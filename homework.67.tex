% -*- fill-column: 110 -*-
\documentclass[12pt]{article}
\usepackage[margin=1in]{geometry}
\usepackage{hyperref}
\usepackage{amsmath}
\usepackage{amsfonts}
\usepackage{amssymb}
\usepackage{amsthm}
\usepackage[margin=1in]{geometry}
% \usepackage{apacite}
\usepackage{color}
%\usepackage{sagetex}
\usepackage{fancyhdr}
\usepackage{setspace}
\pagestyle{fancy}
\lhead{\footnotesize Exercise Sets 5.1}
\rhead{\footnotesize October 11 2013 -- MA325 -- Shae Erisson}

\newcommand*\diff{\mathop{}\!\mathrm{d}}
\begin{document}

% Section 7.3: 16, 17, 18*, 19
% Section 7.2: 4, 5, 6, 15, 16*, 36, 37
% Section 6.2: 1, 7, 8, 9*, 12, 13, 16, 25, 34*
% Section 6.1: 3, 5, 9. 16*, 19, 22

\setcounter{section}{6}
\section*{Exercise Set 6.1 - Set Theory, Element Method of Proof}
\begin{enumerate}
% Section 6.1: 3, 5, 9. 16*, 19, 22
  \setcounter{enumi}{2}
\item % 3
  \setcounter{enumi}{4}
\item % 5
  \setcounter{enumi}{8}
\item % 9
  \setcounter{enumi}{15}
\item % 16*
Let $A = \{a,b,c\}, B = \{b,c,d\}$ and $C = \{b,c,e\}$.
\begin{enumerate}
\item Find $A \cup (B \cap C), (A \cup B) \cap C$ and $(A \cup B) \cap (A \cup C)$. Which of these sets are
  equal?

$A \cup (B \cap C)$ is $\{a,b,c\}$.

$(A \cup B) \cap C$ is $\{b,c\}$.

$(A \cup B) \cap (A \cup C)$ is $\{a,b,c\}$.

The first and third of these sets are equal.
\item Find $A \cap (B \cup C), (A \cap B) \cup C$, and $(A \cap B) \cup (A \cap C)$. Which of these sets are
  equal?

$A \cap (B \cup C)$ is $\{b,c\}$.

$(A \cap B) \cup C$ is $\{b,c,e\}$

$(A \cap B) \cup (A \cap C)$ is $\{b,c\}$.

The first and third of these are equal.

\end{enumerate}
  \setcounter{enumi}{18}
\item % 19
  \setcounter{enumi}{21}
\item % 22
\end{enumerate}
\section*{Exercise Set 6.2 - Sequences}
% Section 6.2: 1, 7, 8, 9*, 12, 13, 16, 25, 34*
\begin{enumerate}
\item % 1
\setcounter{enumi}{6}
\item % 7
\item % 8
\item % 9*
For all sets $A$, $B$, and $C$, 
$$(A - B) \cup (C - B) = (A \cup C) - B$$
We start with $(A - B) \cup (C - B) \subseteq (A \cup C) - B$.

Let $x \in (A - B)$. We know that $x \not\in B$ by definition of set difference. Since $x \in A$ and $x
\not\in B$, we know $x \in (A \cup C) - B$.

Let $x \in (C - B)$. We know that $x \not\in B$ by definition of set difference. Since $x \in C$ and $x
\not\in B$, we know $x \in (A \cup C) - B$.

Thus we have shown that $(A - B) \cup (C - B) \subseteq (A \cup C) - B$.

We continue with $(A - B) \cup (C - B) \supseteq (A \cup C) - B$.

Let $x \in A$ and $x \not\in C$. We know that $x \in (A \cup C)$ by definition of union, and since $x \not\in
B$, we know that $(A - B) \cup (C - B) \supseteq (A \cup C) - B$.

Thus we have shown $(A - B) \cup (C - B) = (A \cup C) - B$.

\setcounter{enumi}{11}
\item % 12
\item % 13
\setcounter{enumi}{15}
\item % 16
\setcounter{enumi}{24}
\item % 25
\setcounter{enumi}{33}
\item % 34*

\end{enumerate}
\section*{Exercise Set 7.2 - One-to-One and Onto, Inverse Functions}
% Section 7.2: 4, 5, 6, 15, 16*, 36, 37
\begin{enumerate}
\setcounter{enumi}{3}
\item % 4
\item % 5
\item % 6
\setcounter{enumi}{14}
\item % 15
\item % 16*
$f(x) = \dfrac{x}{x^{2}+1}$ is not surjective, because it does produce all real numbers.
\setcounter{enumi}{35}
\item % 36
\item % 37
\end{enumerate}

\section*{Exercise Set 7.3 - Composition of Functions}
% Section 7.3: 16, 17, 18*, 19
\begin{enumerate}
\setcounter{enumi}{15}
\item % 16
\item % 17
\end{enumerate}
\end{document}

% LocalWords: LocalWords differentiable
