% -*- fill-column: 110 -*-
\documentclass[12pt]{article}
\usepackage[margin=1in]{geometry}
\usepackage{hyperref}
\usepackage{amsmath}
\usepackage{amsfonts}
\usepackage{amssymb}
\usepackage{amsthm}
\usepackage[margin=1in]{geometry}
% \usepackage{apacite}
\usepackage{color}
%\usepackage{sagetex}
\usepackage{fancyhdr}
\usepackage{setspace}
\pagestyle{fancy}
\lhead{\footnotesize Exercise Sets 6.1,6.2,7.2,7.3}
\rhead{\footnotesize November 6 2013 -- MA325 -- Shae Erisson}

\newcommand*\diff{\mathop{}\!\mathrm{d}}
\newenvironment{modenumerate}
  {\enumerate\setupmodenumerate}
  {\endenumerate}

\newif\ifmoditem
\newcommand{\setupmodenumerate}{%
  \global\moditemfalse
  \let\origmakelabel\makelabel
  \def\moditem##1{\global\moditemtrue\def\mesymbol{##1}\item}%
  \def\makelabel##1{%
    \origmakelabel{##1\ifmoditem\rlap{\mesymbol}\fi\enspace}%
    \global\moditemfalse}%
}

\begin{document}

% Section 7.3: 16, 17, 18*, 19
% Section 7.2: 4, 5, 6, 15, 16*, 36, 37
% Section 6.2: 1, 7, 8, 9*, 12, 13, 16, 25, 34*
% Section 6.1: 3, 5, 9. 16*, 19, 22

\setcounter{section}{6}
\section*{Exercise Set 6.1 - Set Theory, Element Method of Proof}
\begin{modenumerate}
% Section 6.1: 3, 5, 9. 16*, 19, 22
  \setcounter{enumi}{2}
\item % 3
Let sets $R, S,$ and $T$ be defined as follows:\\
$R = \{x \in \mathbb{Z} | x$ is divisible by $2\}$
$S = \{y \in \mathbb{Z} | y$ is divisible by $3\}$
$T = \{z \in \mathbb{Z} | z$ is divisible by $6\}$
\begin{enumerate}
\item Is $R \subseteq T$? Yes, because everything divisible by 6 is also divisible by 2.
\item Is $T \subseteq R$? No, not all numbers divisible by 2 are also divisible by 6.
\item Is $T \subseteq S$? Yes, all numbers divisible by 6 are also divisible by 3.
\end{enumerate}
  \setcounter{enumi}{4}
\item % 5
Let $C = \{n \in \mathbb{Z} | n = 6r - 5$ for some integer $r\}$ and $D = \{m \in \mathbb{Z} | m = 3s + 1$ for
some integer $s \}$. Prove or disprove the following statements
\begin{enumerate}
\item Is $C \subseteq D$?  % Yes, because $6r-5$ is contains all values produced by $3s + 1$.
Suppose $n$ is any element of $C$. Then $n = 6r - 5$ for some integer $r$. Let $s = 2r - 2$. Then $s$ is an
integer since product and difference of integers is an integer.
$$3s + 1 = 3(2r-2)+1 = 6r-5$$
which equals $n$. Thus $n$ satisfies the condition for being in $D$. Hence every element in $C$ is in $D$.
\item Is $D \subseteq C$?  No, because $3s+1$ produces values that are not produced by $6r-5$.
\end{enumerate}

  \setcounter{enumi}{8}
\item % 9
Complete the following sentences without using the symbols $\cup, \cap, or -$.
\begin{enumerate}
\item $x \not\in A \cup B$ if, and only if, $x$ is not an element of $A$ and x is not an element of $B$.
\item $x \not\in A \cap B$ if, and only if, $x$ is an element of $A$ but not an element of $B$, or if $x$ is
  an element of $B$ but not an element of $A$.
\item $x \not\in A - B$ if, and only if $x$ is an element of $A$ and $x$ is an element of $B$, or if $x$ is an
  element of $B$ and not an element of $A$, or if $x$ is not an element of $A$ and $x$ is not an element of
  $B$.
\end{enumerate}
  \setcounter{enumi}{15}
\moditem{*} % 16*
Let $A = \{a,b,c\}, B = \{b,c,d\}$ and $C = \{b,c,e\}$.
\begin{enumerate}
\item Find $A \cup (B \cap C), (A \cup B) \cap C$ and $(A \cup B) \cap (A \cup C)$. Which of these sets are
  equal?

$A \cup (B \cap C)$ is $\{a,b,c\}$.

$(A \cup B) \cap C$ is $\{b,c\}$.

$(A \cup B) \cap (A \cup C)$ is $\{a,b,c\}$.

The first and third of these sets are equal.
\item Find $A \cap (B \cup C), (A \cap B) \cup C$, and $(A \cap B) \cup (A \cap C)$. Which of these sets are
  equal?

$A \cap (B \cup C)$ is $\{b,c\}$.

$(A \cap B) \cup C$ is $\{b,c,e\}$

$(A \cap B) \cup (A \cap C)$ is $\{b,c\}$.

The first and third of these are equal.

\end{enumerate}

\end{modenumerate}
\section*{Exercise Set 6.2 - Sequences}
% Section 6.2: 1, 7, 8, 9*, 12, 13, 16, 25, 34*
\begin{modenumerate}

\moditem{*} % 9*
For all sets $A$, $B$, and $C$, 
$$(A - B) \cup (C - B) = (A \cup C) - B$$
We start with $(A - B) \cup (C - B) \subseteq (A \cup C) - B$.

Let $x \in (A - B)$. We know that $x \not\in B$ by definition of set difference. Since $x \in A$ and $x
\not\in B$, we know $x \in (A \cup C) - B$.

Let $x \in (C - B)$. We know that $x \not\in B$ by definition of set difference. Since $x \in C$ and $x
\not\in B$, we know $x \in (A \cup C) - B$.

Thus we have shown that $(A - B) \cup (C - B) \subseteq (A \cup C) - B$.

We continue with $(A - B) \cup (C - B) \supseteq (A \cup C) - B$.

Let $x \in A$ and $x \not\in C$. We know that $x \in (A \cup C)$ by definition of union, and since $x \not\in
B$, we know that $(A - B) \cup (C - B) \supseteq (A \cup C) - B$.

Thus we have shown $(A - B) \cup (C - B) = (A \cup C) - B$.


\end{modenumerate}
\section*{Exercise Set 7.2 - One-to-One and Onto, Inverse Functions}
% Section 7.2: 4, 5, 6, 15, 16*, 36, 37
\begin{modenumerate}

\moditem{*} % 16*
$f(x) = \dfrac{x}{x^{2}+1}$ is not surjective, because it does produce all real numbers.
\setcounter{enumi}{35}
\end{modenumerate}

\section*{Exercise Set 7.3 - Composition of Functions}
% Section 7.3: 16, 17, 18*, 19
\begin{modenumerate}
\setcounter{enumi}{15}
\moditem{*} % 18*
If $f : X \rightarrow Y$ and $g : Y \rightarrow Z$ are functions and $g \circ f$ is one-to-one, must $f$ be
one-to-one?

Given $x_{1}$ and $x_{2}$ in $X$, if $f(x_{1}) = f(x_{2})$ then $(g \circ f)(x_{2}) = (g \circ f)(x_{2})$. The
composition of two functions is one-to-one if and only if the two composed functions are both one-to-one.

Thus, $f$ must be one-to-one.
\end{modenumerate}
\end{document}

% LocalWords: LocalWords differentiable
