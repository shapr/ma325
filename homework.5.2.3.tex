% -*- fill-column: 110 -*-
\documentclass[12pt]{article}
\usepackage[margin=1in]{geometry}
\usepackage{hyperref}
\usepackage{amsmath}
\usepackage{amsfonts}
\usepackage{amssymb}
\usepackage{amsthm}
\usepackage[margin=1in]{geometry}
% \usepackage{apacite}
\usepackage{color}
%\usepackage{sagetex}
\usepackage{fancyhdr}
\usepackage{setspace}
\pagestyle{fancy}
\lhead{\footnotesize Exercise Sets 5.1}
\rhead{\footnotesize October 11 2013 -- MA325 -- Shae Erisson}

\newcommand*\diff{\mathop{}\!\mathrm{d}}
\begin{document}

\setcounter{section}{5}
\section*{Exercise Set 5.2 - Mathematical Induction I}
\begin{enumerate}
% Section 5.3: 4, 9*, 16, 17*, 31
  \setcounter{enumi}{3}
\item % 4
  For each integer $n$ with $n \geq 2$ let $P(n)$ be the formula\\
  $\displaystyle\sum_{i=1}^{n-1}i(i+1)=\dfrac{n(n-1)(n+1)}{3}$
\begin{enumerate}
\item $P(1)$ is $1(1+1) = \dfrac{2(2-1)(2+1)}{3}$ which becomes $2 = \dfrac{2(3)}{3}$ which is true.
\item $P(k) = \dfrac{k(k-1)(k+1)}{3}$
\item $P(k+1) = \dfrac{(k+1)(k)(k+2)}{3}$
\item In the inductive step, it must be shown that if $P(k)$ is true, then $P(k+1)$ is true. This is done by
  using the inductive hypothesis.
\end{enumerate}
  \setcounter{enumi}{8}
\item % 9*
  Prove by induction that for all integers $n \geq 3$, $4^{3}+4^{4}+4^{5}+ ... + 4^{n} =
  \dfrac{4(4^{n}-16)}{3}$.

  For our basis step, we check to see if the property holds true for $n = 3$. Filling in $n$ we get $4^{3} =
  \dfrac{4(4^{3}-16)}{3}$. That reduces to $64 = \dfrac{4(64-16)}{3}$ and then $64 = \dfrac{4(48)}{3}$
  and $64 = \dfrac{192}{3}$ with our final step being $64 = 64$.

  For the inductive step: Suppose that $P(k)$ is true for all integers $k \geq 3$.
  We show that $P(k+1) = \dfrac{4(4^{k+1}-16)}{3}$.
  $$4^{3} + ... + 4^{n} + 4^{n+1} = \dfrac{4(4^{k+1}-16)}{3}$$
  $$4^{3} + ... + 4^{n} + 4^{n+1} = \dfrac{4(4 \cdot 4^{k}-16)}{3}$$
  $$4^{3} + ... + 4^{n} + 4^{n+1} = \dfrac{4\cdot4( 4^{k}-4)}{3}$$
  XXX wtf?
  \setcounter{enumi}{15}
\item % 16
Prove by induction that for all integers $n \geq 2$ 
$$ (1- \dfrac{1}{2^{2}})(1-\dfrac{1}{3^{2}}) ... (1-\dfrac{1}{n^{2}}) = \dfrac{n+1}{2n}$$

\end{enumerate}
% Section 5.2: 2, 3, 6, 9, 12*, 14

\end{document}

% LocalWords: LocalWords differentiable
