% -*- fill-column: 110 -*-
\documentclass[12pt]{article}
\usepackage[margin=1in]{geometry}
\usepackage{hyperref}
\usepackage{amsmath}
\usepackage{amsfonts}
\usepackage{amssymb}
\usepackage{amsthm}
\usepackage[margin=1in]{geometry}
% \usepackage{apacite}
\usepackage{color}
%\usepackage{sagetex}
\usepackage{fancyhdr}
\usepackage{setspace}
\pagestyle{fancy}
\lhead{\footnotesize Exercise Sets 5.1}
\rhead{\footnotesize October 11 2013 -- MA325 -- Shae Erisson}

\newcommand*\diff{\mathop{}\!\mathrm{d}}
\begin{document}

\setcounter{section}{5}
\section*{Exercise Set 5.1 - Sequences}
\begin{enumerate}
% Section 5.1: 15,16,19,39,50*,78
  \setcounter{enumi}{14}
\item % 15
For the sequence 0, $-
\dfrac{1}{2}$,$\dfrac{2}{3}$,$-\dfrac{3}{4}$,$\dfrac{4}{5}$,$-\dfrac{5}{6}$,
$\dfrac{6}{7}$, we find the sequence $a_{k} = \dfrac{k-1 \cdot (-1)^{k+1}  }{k}$
\item % 16
For the sequence 3,6,12,24,48,96, we find the sequence $a_{k} =
2^{k-1} * 3$
% 1 = 3 (k * 3)
% 2 = 6 (k * 3)
% 3 = 12 (k * 4?)
% 4 = 24 (k * 6?)
% 5 = 48 (k * 9.6 ??)
% 6 = 96 (k * 16)
% k = 2k (k + 3)
% 2(k-1)
  \setcounter{enumi}{18}
\item % 19
The sum of $\displaystyle\sum_{k=1}^{5}(k+1)$ is 2+3+4+5+6 or 20.
  \setcounter{enumi}{38}
\item % 39
The final term of $\displaystyle\sum_{m=1}^{n+1}m(m+1)$ can be
separated off, giving $\displaystyle\sum_{m=1}^{n}m(m+1) + \displaystyle\sum_{n}^{n+1}m(m+1)$
  \setcounter{enumi}{49}
\item % 50*

  The sequence $\dfrac{1}{2!} + \dfrac{2}{3!} + \dfrac{3}{4!} + ... + \dfrac{n}{(n+1)!}$ can be written as
  this summation $\displaystyle\sum_{k=1}^{n}\dfrac{n}{(n+1)!}$ Since the sum starts at 1, we put that on the
  bottom of the summation for the starting point. The sum ends at n, so we put that on the top of the sum
  symbol. Each separate value is generated with the current value of $n$ substituted into $\dfrac{n}{(n+1)!}$
  so that goes to the right of the sum symbol.

  \setcounter{enumi}{77}
\item % 78

  Prove that for all nonnegative integers $n$ and $r$ with $r + 1 \leq n$, $\displaystyle{n \choose r + 1} =
  \dfrac{n - r}{r+1}\displaystyle{n \choose r}$.

  By definition, $\displaystyle{n \choose r} = \dfrac{n!}{r!(n - r)!}$, for integers $n$ and $r$ with $0 \leq
  r \leq n$.

Thus $\displaystyle{n \choose r + 1}$ will be equal to $\dfrac{n!}{(r+1)!(n - (r+1))!}$ or
$\dfrac{n!}{(r+1)!(n - r -1)!}$.
  
By definition $n!$ is $n \cdot (n - 1) \ldots 3 \cdot 2 \cdot 1$.

Thus, $(r+1)!$ can be rewritten as $r!(r +1)$ and substituted into our earlier expression.

Now our expression is $\dfrac{n!}{(r+1)r!(n - r)!}$, which gets us to $\dfrac{1}{r+1} \cdot
\dfrac{n!}{r!(n - r + 1)! }$.

We know our goal for the bottom half of the fraction is $r!(n-r)!$ and we have $r!(n - r + 1)!$.

By definition of factorial, we know that $(n - r + 1)$ is the same as  by $(n - r + 1)(n -r)$ so we
split that out into the term by $\dfrac{1}{n-r+1}$.

We then have $\dfrac{1}{r+1} \cdot \dfrac{n!}{r!(n - r)! } \cdot \dfrac{1}{n-r+1}$, at which point we become
stuck on this problem.

\end{enumerate}
\end{document}

% LocalWords: LocalWords differentiable
