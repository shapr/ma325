% -*- fill-column: 110 -*-
\documentclass[12pt]{article}
\usepackage[margin=1in]{geometry}
\usepackage{hyperref}
\usepackage{amsmath}
\usepackage{amsfonts}
\usepackage{amssymb}
\usepackage{amsthm}
\usepackage[margin=1in]{geometry}
% \usepackage{apacite}
\usepackage{color}
%\usepackage{sagetex}
\usepackage{fancyhdr}
\usepackage{setspace}
\pagestyle{fancy}
\lhead{\footnotesize Exercise Sets 4.1,4.2,4.3,4.4,4.6}
\rhead{\footnotesize September 20 2013 -- MA325 -- Shae Erisson}

\newcommand*\diff{\mathop{}\!\mathrm{d}}
\begin{document}

% Section 4.6: 11, 13, 28*
\setcounter{section}{4}
\section*{Exercise Sets 4.6, 4.7}

\subsection{Direct Proof and Counterexample: Division into Cases and the Quotient Remainder Theorem} 
% Section 4.6: 11, 13, 28
\begin{enumerate}
\setcounter{enumi}{10} % 11
\item % 

\setcounter{enumi}{12} % 13
\item %

\setcounter{enumi}{27} % 28
\item % for all integers m and n, if mn is even then m is even or n is even
  \begin{proof}[For all integers $m$ and $n$, if $mn$ is even, then $m$ is even or $n$ is even.]
    % I think we want to cast this as a mod 2 = 0 problem
    Suppose not. That is, suppose there is an integer $mn$ that is even, and neither $m$ nor $n$ are even.
That is, neither $m$ nor $n$ have 2 as a factor.
By the Unique Factorization Theorem, either $n$ or $m$ must have 2 as a factor in order for $mn$ to be
even. Since neither $m$ nor $n$ have 2 as a factor, this is a contradiction.
  \end{proof}

  \begin{proof}[For all integers $m$ and $n$ if $mn$ is even, then $m$ is even or $n$ is even.]
    We proceed by contrapositive. Our original statement is $\forall m,n \in \mathbb{Z}$ where $mn \mod{2} = 0
    \rightarrow n \mod{2} = 0 \vee m \mod{2} = 0$. The contrapositive is $\forall m,n \in \mathbb{Z}$ where
    ... The rest of this homework has been canceled due to life issues.
  \end{proof}

\end{enumerate}

\end{document}

% LocalWords: LocalWords differentiable
