% -*- fill-column: 110 -*-
\documentclass[12pt]{article}
\usepackage[margin=1in]{geometry}
\usepackage{hyperref}
\usepackage{amsmath}
\usepackage{amsfonts}
\usepackage{amssymb}
\usepackage{amsthm}
\usepackage[margin=1in]{geometry}
% \usepackage{apacite}
\usepackage{color}
%\usepackage{sagetex}
\usepackage{fancyhdr}
\usepackage{setspace}
\pagestyle{fancy}
\lhead{\footnotesize Exercise Sets 5.1}
\rhead{\footnotesize October 11 2013 -- MA325 -- Shae Erisson}

\newcommand*\diff{\mathop{}\!\mathrm{d}}
\newenvironment{modenumerate}
  {\enumerate\setupmodenumerate}
  {\endenumerate}

\newif\ifmoditem
\newcommand{\setupmodenumerate}{%
  \global\moditemfalse
  \let\origmakelabel\makelabel
  \def\moditem##1{\global\moditemtrue\def\mesymbol{##1}\item}%
  \def\makelabel##1{%
    \origmakelabel{##1\ifmoditem\rlap{\mesymbol}\fi\enspace}%
    \global\moditemfalse}%
}


\begin{document}

% Section 10.1: 1,3,15,27*,31,37

\setcounter{section}{10}
\section*{Exercise Set 10.1} % - Graphs, Definitions, and Basic Properties}
\begin{modenumerate}
\item % 1
  A graph consists of two finite sets: a nonempty set $V(G)$ of vertices and a set $E(G)$ of edges, where each
  edge is associated with a set consisting of connected edges.  \setcounter{enumi}{2}
\item % 3
Two distinct edges in a graph are parallel if, and only if, they have the exact same set of endpoints.
\setcounter{enumi}{14}
\item % 15
The handshake theorem says that the total degree of a graph is twice the number of edges in the graph.
\setcounter{enumi}{26}
\moditem{*} % 27
\begin{enumerate}
\item 
For each person in a 15 person group to have exactly three friends, each vertex would require three edges that
go to other vertices. If each point had three edges, the total degree of the graph would be $\dfrac{3(15)}{2}$
or $\dfrac{45}{2}$. Since this is not an integer, there is no way for 15 people to each have 3 friends in a
group.
\item Yes, it is possible for a group of 4 people to have three friends each. The resulting graph would be
  fully connected, where each vertex is connected to the other three by an edge representing friendship.
\end{enumerate}

\setcounter{enumi}{30}
\item % 31
\begin{proof}[Prove that the sum of an odd number of odd numbers is odd.]

We proceed by induction.

Let the series $a_{1},a_{2} \dots a_{n}$ be a series of odd numbers.

Let the property $P(n)$ be $\displaystyle\sum_{m=1}^{n=2k+1}a_{i}$.

Base Case: $\displaystyle\sum_{m=1}^{n=1}a_{1}$ The 'sum' of one odd number is odd.

Inductive Hypothesis: $\displaystyle\sum_{m=1}^{n=2k+1}a_{1}$ will be odd.

Inductive Step: $\displaystyle\sum_{m=1}^{n=2k+2}a_{1}$ will be odd.

By assumption we know that the sum from 1 to $2k+1$ is odd. Thus we are adding two more odd numbers to the
sum. We know that the sum of any two odd numbers is even, and we know that the sum of an odd number and an
even number is odd.
\end{proof}
\setcounter{enumi}{36}
\item % 37
Which of the following graphs are bipartite?
\begin{enumerate}
\item This graph is bipartite.
\item This graph is not bipartite.
\item This graph is bipartite.
\item This graph is not bipartite.
\item This graph is bipartite.
\item This graph is not bipartite.
\end{enumerate}
\end{modenumerate}
\setcounter{section}{10}
\section{10.5}
\begin{modenumerate}
\setcounter{enumi}{2}
\item % 3
  The total degree of a tree with $n$ vertices is $2(n-1)$ because, by definition, a tree is circuit-free and
  connected. If a graph is connected, there will be at least one edge to every vertex. If a graph is
  circuit-free, there will be at most one edge between any two chosen points.
  \begin{proof}[For any positive integer $n$, any tree with $n$ vertices has $n-1$ edges.]
    Let our property $P(n)$ be the sentence

    Any tree with $n$ vertices has $n-1$ edges.

    We will use induction to show that this is true for any number of vertices $n\geq 1$.
    
    Base Case: Let $T$ be any tree with one vertex. Then $T$ has zero edges (since it contains no loops). But
    $0 = 1-1$ so $P(1)$ is true.

    Inductive Hypothesis: We assume that for all integers $k \geq 1$, $P(k)$ is true.

    Inductive Step: If $P(k)$ is true, $P(k+1)$ is true.

    Suppose $k$ is any positive integer for which $P(k)$ is true. In other words, suppose that:

    Any tree with $k$ vertices has $k -1$ edges.

    We want to show that $P(k+1)$ is true, in other words we want to show that 

    Any tree with $k+1$ vertices has $(k+1)-1 = k$ edges.

    Let $T$ be a particular but arbitrarily chosen tree with $k + 1$ vertices. Since $k$ is a positive
    integer, $(k+1) \geq 2$, and so $T$ has more than one vertex. By the next proof, $T$ has a vertex $v$ of
    degree 1. Also, since $T$ has more than one vertex, there is at least one other vertex in $T$ besides
    $v$. Thus there is an edge $e$ connecting $v$ to the rest of $T$. 

    We define a subgraph $T'$ of $T$ so that 

    $V(T') = V(T) - \{v\}$

    then

    $E(T') = E(T) - \{e\}$
    \begin{enumerate}
    \item The number of vertices of $T'$ is $(k+1)-1=k$.
    \item $T'$ is circuit-free
    \item $T'$ is connected
    \end{enumerate}
    Hence, by definition of tree, $T'$ is a tree. Since $T'$ has $k$ vertices, by inductive hypothesis

    the number of edges of $T' = $(the number of vertices of $T') - 1$.

    $= k - 1$

    But then,

    the number of edges of $T = ($the number of edges of $T') + 1$

    $= (k-1)+1$

    $= k$
  \end{proof}

  \begin{proof}[Any tree that has more than one vertex has at least one vertex of degree 1]
    Let $T$ be a particular but arbitrarily chosen tree that has more than one vertex, and consider the
    following algorithm:
    \begin{enumerate}
    \item Pick a vertex $v$ of $T$ and let $e$ be an edge incident on $v$.
    \item While degree$($v$) > 1$ repeat steps i,ii, and iii.
      \begin{enumerate}
      \item Choose $e'$ to be an edge incident on $v$ such that $e' \neq e$
      \item Let $v'$ be the vertex at the other end of $e'$ from $v$.
      \item Let $e = e'$ and $v = v'$.
      \end{enumerate}
    \end{enumerate}
    The algorithm described must eventually terminate because the set of vertices of the tree $T$ is finite,
    and $T$ is circuit-free. When it does, a vertex $v$ of degree 1 will have been found.
  \end{proof}
\end{modenumerate}
\end{document}

% LocalWords: LocalWords differentiable
