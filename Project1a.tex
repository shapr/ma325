% -*- fill-column: 110 -*-
\documentclass[12pt]{article}
\usepackage[margin=1in]{geometry}
\usepackage{hyperref}
\usepackage{amsmath}
\usepackage{amsfonts}
\usepackage{amssymb}
\usepackage{amsthm}
\usepackage[margin=1in]{geometry}
% \usepackage{apacite}
\usepackage{color}
%\usepackage{sagetex}
\usepackage{fancyhdr}
\usepackage{setspace}
\pagestyle{fancy}
\lhead{\footnotesize Project 1}
\rhead{\footnotesize October 4 2013 -- MA325 -- Shae Erisson, Jordan Herring}

\newcommand*\diff{\mathop{}\!\mathrm{d}}
\begin{document}

\section*{For all integers $n$,$[n+(n+1)+(n+2)+(n+3)]$ mod 4 = 2}
\begin{proof} % For all integers $n$,$\[n+(n+1)+(n+2)+(n+3)\]$ mod 4 = 2]
By definition, $n$ mod $d$ = $r$, can be rewritten as $n = dq + r$.

%Thus we wish to prove that $[n+(n+1)+(n+2)+(n+3)]$ mod 4 = 2 is equal to 
We start with $[n+(n+1)+(n+2)+(n+3)]$ and rewrite it as $4n+6$.
We can then use the given definition of mod to show that our goal is
to prove that $4n + 6 = 4q+2$ for some value of $q$.
We simplify the equation to $2n + 3 = 2q + 1$ and then reorder to get
$2n + 2 = 2q$ and then we factor out a 2 to get $n + 1 = q$. We then substitute our value of
$q$ into the equation above. This gives us $4(n +1) +6 = 4(n+1) + 2$
We simplify this to $4n + 4 + 4 = 4n + 4$.
This simplifies to $n + 2 = n + 1$, thus disproving this equation.
\end{proof}
\end{document}

% LocalWords: LocalWords differentiable
