% -*- fill-column: 110 -*-
\documentclass[12pt]{article}
\usepackage[margin=1in]{geometry}
\usepackage{hyperref}
\usepackage{amsmath}
\usepackage{amsfonts}
\usepackage{amssymb}
\usepackage{amsthm}
\usepackage[margin=1in]{geometry}
% \usepackage{apacite}
\usepackage{color}
%\usepackage{sagetex}
\usepackage{fancyhdr}
\usepackage{setspace}
\pagestyle{fancy}
\lhead{\footnotesize Exercise Sets 4.1,4.2,4.3,4.4,4.6}
\rhead{\footnotesize September 20 2013 -- MA325 -- Shae Erisson}

\newcommand*\diff{\mathop{}\!\mathrm{d}}
\begin{document}

% Section 4.6: 11, 13, 28*
% Section 4.4: 17, 19
% Section 4.3: 20*, 27
% Section 4.2: 16, 17*
% Section 4.1: 10, 12, 30, 31, 41, 58*

\setcounter{section}{4}
\section*{Exercise Sets 4.1, 4.2, 4.3, 4.4, 4.6}
\subsection{Direct Proof and Counterexample}
\begin{enumerate}
% Section 4.1: 10,12,30,31,58*
  \setcounter{enumi}{9}
\item % 10
For the formula $2n^{2}-5n+2$, if $n$ is 3, the result is 5. The
number 5 is prime because it has no integer divisors other than itself
and 1.
  \setcounter{enumi}{11}
\item % 12
The smallest counterexample can be found by squaring two and adding
one. Thus the number five is the smallest counterexample.
  \setcounter{enumi}{29}
\item % 30
\begin{proof}
  $\forall m \in \mathbb{Z}$ if m is an even number, then $3m + 5$ is
  odd.
  From our definitions, an even number $n$ is $n = 2k$. We can
  substitute this in to get $3(2k) + 5$ which is equivalent to $2(3k)
  + 5$.
  $2(3k)$ is an even number by our earlier definition, and $5$ is
  equivalent to $2(2)+1$ which is an odd number by our earlier
  definition.
  As we proved earlier, an even number plus an odd number is an odd number.
\end{proof}
\item % 31
  \begin{proof}
      $\forall k \in \mathbb{Z}$ if $k$ is odd and $m$ is even, $k^{2} + m^{2}$ is odd.
      According to our definitions, an odd integer is $2k + 1$ and an even integer is $2j$.
      $(2k + 1)^{2}$ is $4k^{2} + 4k + 1$ or $2(2k^{2}+2k) + 1$ which is by definition an odd integer.
      $2k^{2}$ is $2(k^{2})$ which is by definition an even integer.
By our earlier proof that the sum of an even integer and an odd integer results in an odd integer, we can show
that the sum of a squared even integer and a squared odd integer will be odd.
  \end{proof}
  \setcounter{enumi}{57}
\item % 58*
  \begin{proof}
    The difference of the squares of any two consecutive integers is odd.
    $2k$ defines an even integer and $2k + 1$ defines an odd integer. If we assume that the $k$ is the same
    integer in both, we can see prove that consecutive integers will be one even and one odd number.
    Per our proof in the previous question we know that the sum of an even and odd squared integer will be odd.
  \end{proof}

\end{enumerate}

\subsection{Direct Proof and Counterexample} % % Section 4.2: 16, 17*
\begin{enumerate}
\setcounter{enumi}{15}
\item % 4.2 16
  \begin{proof}[The product of any two rational numbers is a rational number.]

  \end{proof}

\item % 4.2 17*
  \begin{proof}[The difference of any two rational numbers is a rational number.]
  \end{proof}
\end{enumerate}
\end{document}

% LocalWords: LocalWords differentiable
