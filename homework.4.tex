% -*- fill-column: 110 -*-
\documentclass[12pt]{article}
\usepackage[margin=1in]{geometry}
\usepackage{hyperref}
\usepackage{amsmath}
\usepackage{amsfonts}
\usepackage{amssymb}
\usepackage{amsthm}
\usepackage[margin=1in]{geometry}
% \usepackage{apacite}
\usepackage{color}
%\usepackage{sagetex}
\usepackage{fancyhdr}
\usepackage{setspace}
\pagestyle{fancy}
\lhead{\footnotesize Exercise Sets 4.1,4.2,4.3,4.4,4.6}
\rhead{\footnotesize September 20 2013 -- MA325 -- Shae Erisson}

\newcommand*\diff{\mathop{}\!\mathrm{d}}
\begin{document}

% Section 4.6: 11, 13, 28*
% Section 4.4: 17, 19
% Section 4.3: 20*, 27
% Section 4.2: 16, 17*
% Section 4.1: 10, 12, 30, 31, 41, 58*

\setcounter{section}{4}
\section*{Exercise Sets 4.1, 4.2, 4.3, 4.4, 4.6}
\subsection{Direct Proof and Counterexample : Introduction}
\begin{enumerate}
% Section 4.1: 10,12,30,31,58*
  \setcounter{enumi}{9}
\item % 10
For the formula $2n^{2}-5n+2$, if $n$ is 3, the result is 5. The
number 5 is prime because it has no integer divisors other than itself
and 1.
  \setcounter{enumi}{11}
\item % 12
The smallest counterexample can be found by squaring two and adding
one. Thus the number five is the smallest counterexample.
  \setcounter{enumi}{29}
\item % 30
\begin{proof}
  $\forall m \in \mathbb{Z}$ if m is an even number, then $3m + 5$ is
  odd.
  From our definitions, an even number $n$ is $n = 2k$. We can
  substitute this in to get $3(2k) + 5$ which is equivalent to $2(3k)
  + 5$.
  $2(3k)$ is an even number by our earlier definition, and $5$ is
  equivalent to $2(2)+1$ which is an odd number by our earlier
  definition.
  As we proved earlier, an even number plus an odd number is an odd number.
\end{proof}
\item % 31
  \begin{proof}
      $\forall k \in \mathbb{Z}$ if $k$ is odd and $m$ is even, $k^{2} + m^{2}$ is odd.
      According to our definitions, an odd integer is $2k + 1$ and an even integer is $2j$.
      $(2k + 1)^{2}$ is $4k^{2} + 4k + 1$ or $2(2k^{2}+2k) + 1$ which is by definition an odd integer.
      $2k^{2}$ is $2(k^{2})$ which is by definition an even integer.
By our earlier proof that the sum of an even integer and an odd integer results in an odd integer, we can show
that the sum of a squared even integer and a squared odd integer will be odd.
  \end{proof}
  \setcounter{enumi}{57}
\item % 58*
  \begin{proof}
    The difference of the squares of any two consecutive integers is odd.
    $2k$ defines an even integer and $2k + 1$ defines an odd integer. If we assume that the $k$ is the same
    integer in both, we can see prove that consecutive integers will be one even and one odd number.
    Per our proof in the previous question we know that the sum of an even and odd squared integer will be odd.
  \end{proof}

\end{enumerate}

\subsection{Direct Proof and Counterexample : Rational Numbers} % Section 4.2: 16, 17*
\begin{enumerate}
\setcounter{enumi}{15}
\item % 4.2 16
  \begin{proof}[The product of any two rational numbers is a rational number.]
Let $r$ and $s$ be rational numbers. Then there are integers $a,b,c,d$ such that $r = \dfrac{a}{b}$ and $s =
\dfrac{c}{d}$ with $b,d \neq 0$. So, we can multiply $r s$ to get $r \cdot s = \dfrac{a}{b} \cdot \dfrac{c}{d}
= \dfrac{ac}{db}$ Since the product of two integers is an integer, $a \cdot c$ and $b \cdot d$ are
integers. According to the zero product property we know that if $b$ and $d$ are not zero, their product will
not be zero. Since we know that $b$ and $d$ are not zero, their product will not be zero. Therefore $r \cdot
s$ is a rational number.
  \end{proof}

\item % 4.2 17*
  \begin{proof}[The difference of any two rational numbers is a rational number.]
Let $r$ and $s$ be rational numbers. Then there are integers $a,b,c,d$ such that $r = \dfrac{a}{b}$ and $s =
\dfrac{c}{d}$ with $b,d \neq 0$. So, we can subtract $r s$ to get $r - s = \dfrac{a}{b} - \dfrac{c}{d}
= \dfrac{a - c}{d - b}$ Since the difference of two integers is an integer, $a - c$ and $b - d$ are
integers. Even so, we cannot be sure that $c - d$ will never be zero, and rational integers require that the
denominator be non-zero. Thus the difference of any two rational numbers is not always a rational number.
  \end{proof}
\end{enumerate}

\subsection{Direct Proof and Counterexample: Divisibility} % Section 4.3: 20*, 27
\begin{enumerate}
\setcounter{enumi}{19} % 20
\item % the sum of any three consecutive integers is divisible by 3
  Three consecutive integers are expressed as $k, k+1, k+2$ by definition. We express this as a
  fraction. $\dfrac{k + k + 1 + k + 2}{3}$ We can algebraically rearrange this as $\dfrac{k + k + k + 1 +
    2}{3}$ and then simplify to $\dfrac{3k + 3}{3}$. By dividing the entire fraction by 3, this can then be
  further simplified to $k + 1$, showing that the sum of three consecutive integers is divisible by 3.
  \setcounter{enumi}{26} % 27
\item % for all integers a, b, and c: if a|(b+c) then a|b or a|c
The smallest counterexample for this is $2|4$ but when we break 4 down into $1+3$, we can see that $2|1$
fails, and $2|3$ also fails.
\end{enumerate}

\subsection{Direct Proof and Counterexample: Division into Cases and the Quotient Remainder Theorem} 
% Section 4.4: 17, 19
\begin{enumerate}
\setcounter{enumi}{16} % 17
\item % prove that the product of any two consecutive integers is even
Let the two consecutive integers be $k$ and $k + 1$. The product of these will be $k \cdot k + k$.
In the case that $k$ is an odd number, $k + 1$ will be an even number, and $k \cdot k + 1$ will have 2 as one
of its factors.
In the case that $k$ is an even number, and $k + 1$ is odd, $k \cdot k + 1$ will have 2 as one of its factors.
Since 2 is a factor for both cases, the product of any two consecutive integers must be even.
\setcounter{enumi}{18} % 19
\item The rest of this homework has been canceled due to life issues.% prove that for all integers n, n^2 - n + 3 is odd
\end{enumerate}
\end{document}

% LocalWords: LocalWords differentiable
