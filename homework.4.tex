% -*- fill-column: 110 -*-
\documentclass[12pt]{article}
\usepackage[margin=1in]{geometry}
\usepackage{hyperref}
\usepackage{amsmath}
\usepackage{amsfonts}
\usepackage{amssymb}
\usepackage{amsthm}
\usepackage[margin=1in]{geometry}
% \usepackage{apacite}
\usepackage{color}
%\usepackage{sagetex}
\usepackage{fancyhdr}
\usepackage{setspace}
\pagestyle{fancy}
\lhead{\footnotesize Exercise Sets 4.1,4.2,4.3,4.4,4.6}
\rhead{\footnotesize September 20 2013 -- MA325 -- Shae Erisson}

\newcommand*\diff{\mathop{}\!\mathrm{d}}
\begin{document}

% Section 4.6: 11, 13, 28*
% Section 4.4: 17, 19
% Section 4.3: 20*, 27
% Section 4.2: 16, 17*
% Section 4.1: 10, 12, 30, 31, 41, 58*

\setcounter{section}{4}
\section*{Exercise Sets 4.1, 4.2, 4.3, 4.4, 4.6}
\subsection{Direct Proof and Counterexample}
\begin{enumerate}
% Section 4.1: 10,12,30,31,58*
  \setcounter{enumi}{9}
\item % 10
For the formula $2n^{2}-5n+2$, if $n$ is 3, the result is 5. The
number 5 is prime because it has no integer divisors other than itself
and 1.
  \setcounter{enumi}{11}
\item % 12
The smallest counterexample can be found by squaring two and adding
one. Thus the number five is the smallest counterexample.
  \setcounter{enumi}{29}
\item % 30
\begin{proof}
  
\end{proof}
\item % 31

  \setcounter{enumi}{57}
\item % 58*

\end{enumerate}

\subsection{4.2}
\end{document}

% LocalWords: LocalWords differentiable
