% -*- fill-column: 110 -*-
\documentclass[12pt]{article}
\usepackage[margin=1in]{geometry}
\usepackage{hyperref}
\usepackage{amsmath}
\usepackage{amsfonts}
\usepackage{amssymb}
\usepackage{amsthm}
\usepackage[margin=1in]{geometry}
% \usepackage{apacite}
\usepackage{color}
%\usepackage{sagetex}
\usepackage{fancyhdr}
\usepackage{setspace}
\pagestyle{fancy}
\lhead{\footnotesize Project 2}
\rhead{\footnotesize November 4 2013 -- MA325 -- Shae Erisson, Jordan Herring}

\newcommand*\diff{\mathop{}\!\mathrm{d}}

\newtheorem{case}{Case}
\newtheorem{subcase}{Case}
\numberwithin{subcase}{case}

\begin{document}


\section*{For sets $A$,$B$, and $C$,$(A\cup B) - C = (A - C) \cup (B - C)$}
\begin{proof} % For all integers $n$,$\[n+(n+1)+(n+2)+(n+3)\]$ mod 4 = 2]

For sets $A$,$B$, and $C$,$(A\cup B) - C = (A - C) \cup (B - C)$

To prove set equality, we prove that the left side of the equation is a subset of the right side of the
equation, and that the right side of the equation is a subset of the left side of the equation.\\
The union of $A$ and $B$ is defined as $A \cup B = \{ x \in U | x \in A$ or $x \in B\}$.\\
The difference of two sets $A$ and $B$ is defined as $A - B = \{x \in U | x \in A$ and $x \not\in B\}$.

We start by proving $(A\cup B) - C \subseteq (A - C) \cup (B - C)$ by element chasing.
By definition, $A \cup B$ is equivalent to $x \in A$ or $x \in B$, giving two cases.
\begin{case}
$x \in A$. As $x \in A$, $x \in A \cup B$ by definition of union.
  \begin{subcase}
    $x \not\in C$. As $x \not\in C$, $x \in (A\cup B) - C$ by definition of set difference.
  \end{subcase}
  \begin{subcase}
    $x \in C$. As $x \in C$, $x \not\in (A \cup B) - C$ by definition of set difference.
  \end{subcase}
\end{case}
\begin{case}
  $x \in B$. As $x \in B$, $x \in A \cup B$ by definition of union.
  \begin{subcase}
    $x \not\in C$. As $x \not\in C$, $x \in (A\cup B) - C$ by definition of set difference.
  \end{subcase}
  \begin{subcase}
    $x \in C$. As $x \in C$, $x \not\in (A \cup B) - C$ by definition of set difference.
  \end{subcase}
\end{case}

We continue by proving $(A\cup B) - C \supseteq (A - C) \cup (B - C)$ by element chasing.
\begin{case}
$x \in A$. 
\begin{subcase}
  $x \in C$. As $x \in C$, $x \not\in A - C$ by definition of set difference. Thus $x \not\in (A - C) \cup (B
  - C)$ by definition of union.
\end{subcase}
\begin{subcase}
  $x \not\in C$. As $x \not\in C$, $x \in A - C$ by definition of set difference. Thus $x \in (A - C) \cup (B
  - C)$ by definition of union.
\end{subcase}
\end{case}
\begin{case}
  $x in B$.
  \begin{subcase}
    $x \in C$. As $x \in C$, $x \not \in B - C$ by definition of set difference. Thus $x \not\in (A - C) \cup (B
    - C)$ by definition of union.
  \end{subcase}
  \begin{subcase}
    $x \not \in C$. As $x \not \in C$, $x \in B - C$ by definition of set difference. Thus $x \in (A - C) \cup
    (B - C)$ by definition of union.
  \end{subcase}
\end{case}
\end{proof}
\end{document}

% LocalWords: LocalWords differentiable
