% -*- fill-column: 110 -*-
\documentclass[12pt]{article}
\usepackage[margin=1in]{geometry}
\usepackage{hyperref}
\usepackage{amsmath}
\usepackage{amsfonts}
\usepackage{amssymb}
\usepackage{amsthm}
\usepackage[margin=1in]{geometry}
% \usepackage{apacite}
\usepackage{color}
%\usepackage{sagetex}
\usepackage{fancyhdr}
\usepackage{setspace}
\pagestyle{fancy}
\lhead{\footnotesize Exercise Sets 5.6 5.7}
\rhead{\footnotesize October 32 2013 -- MA325 -- Shae Erisson}

\newcommand*\diff{\mathop{}\!\mathrm{d}}

\newenvironment{modenumerate}
  {\enumerate\setupmodenumerate}
  {\endenumerate}

\newif\ifmoditem
\newcommand{\setupmodenumerate}{%
  \global\moditemfalse
  \let\origmakelabel\makelabel
  \def\moditem##1{\global\moditemtrue\def\mesymbol{##1}\item}%
  \def\makelabel##1{%
    \origmakelabel{##1\ifmoditem\rlap{\mesymbol}\fi\enspace}%
    \global\moditemfalse}%
}

\begin{document}


% Section 5.6: 2, 3, 13, 26, 27, 31*
% Problem 27 has a typo in the book.  The problem should read F_k^2 - F_{k-1}^2 = F_k F_{k+1}-F_{k+1}F_{k-1}.
\setcounter{section}{5}
\subsection{Exercise Set 5.6 - Defining Sequences Recursively}
\begin{modenumerate}

\setcounter{enumi}{1}
\item % 2
Find the first four terms of the sequence $b_{k} = b_{k-1} + 3k$ for all integers $k \geq 2$, where $b_{1} =
1$
\begin{itemize}
\item $b_{1} = 1$
\item $b_{2} = 7$
\item $b_{3} = 10$
\item $b_{4} = 22$
\end{itemize}

\item % 3
Find the first four terms of the sequence $c_{k} = k(c_{k-1})^{2}$ for all integers $k \geq 1$, where
$c_{0}=1$
\begin{itemize}
\item $c_{0} = 1$
\item $c_{1} = 1$
\item $c_{2} = 2$
\item $c_{3} = 6$
\item $c_{4} = 144$
\end{itemize}

\setcounter{enumi}{12}
\item % 13

Let $t_{0},t_{1},t_{2},...$ be defined by the formula $t_{n}=2+n$ for all integers $n \geq 0$. Show that this
sequence satisfies the recurrence relation $t_{k}=2t_{k-1}-t_{k-2}$.

Since $t_{k-1}=2+(n-1)$ and $t_{k-2}=2+(n-2)$ we can substitute those into $t_{k}=2t_{k-1}-t_{k-2}$ to get
$t_{k} = 2(2+(n-1)) - (2+(n-2))$. Combining the integers gets us to $t_{k} = 2(n+1) - n$ and distributing gets
us to $t_{k} = 2n+2 - n$. Once again, combining like terms and rearranging gets us to $t_{k} = 2+n$. Thus
these two forms are equal.

\setcounter{enumi}{25}
\item % 26
  Prove that $F_{k}=3F_{k-3}+2F_{k-4}$ for all integers $k\geq 4$, where F is the Fibonacci sequence.
  \footnote{I'm convinced there's a more general proof hiding here that shows that multiplying two sequential
    lesser indices by two sequential Fibonacci numbers will also give a Fibonacci number.}

The Fibonacci sequence is defined as $F_{k}=F_{k-1}+F_{k-2}$. By our earlier definition of the Fibonacci
sequence, we can combine any two sequential indices to get the number that has an index one greater than the
largest of the two indices we have combined. Thus two values of $F_{k-3}$ and two values of $F_{k-4}$ will
give two values of $F_{k-2}$. We show several steps in this sequence.
  $$F_{k}=3F_{k-3}+2F_{k-4}$$
  $$F_{k}=2F_{k-2}+F_{k-3}$$
  $$F_{k}=F_{k-1}+F_{k-2}$$
  $$F_{k}=F_{k}$$

\item % 27
Prove that $F_k^2 - F_{k-1}^2 = F_k F_{k+1}-F_{k+1}F_{k-1}$ for all integers $k \geq 1$.
% $$F_k^2 - F_{k-1}^2 = F_{k+1}(F_k-F_{k-1})$$
% $$F_k^2 - F_{k-1}^2 = F_{k+1}(F_k-F_{k-1})$$

% $$F_{k}F_k - F_{k-1}F_{k-1} = F_k F_{k+1}-F_{k+1}F_{k-1}$$
% $$F_{k}F_{k-1}F_{k-2} - F_{k-1}F_{k-1} = F_k F_{k+1}-F_{k+1}F_{k-1}$$
% $$F_{k-1}(F_{k}F_{k-2} - F_{k-1}) = F_k F_{k+1}-F_{k+1}F_{k-1}$$
% $$F_{k-1}(F_{k}F_{k-2} - F_{k-1}) = F_{k-1}F_{k-2}F_{k+1}-F_{k+1}F_{k-1}$$
% $$F_{k-1}(F_{k}F_{k-2} - F_{k-1}) = F_{k-1}(F_{k-2}F_{k+1}-F_{k+1})$$
\setcounter{enumi}{30}
\moditem{*} Use mathematical induction to prove that for all integers $n < 1,
F_{n}<2^{n}$

Base Case: $1<2^{1}$ since $1<2$, this is correct.

Inductive Hypothesis: Assume that $F_{n}<2^{n}$ for all integers $j \geq 1$, up to some integer $k \in
\mathbb{Z}$.

Inductive Step: We want to show that $F_{n+1}<2^{n+1}$.

\end{modenumerate}
% \subsection{Exercise Set 5.7 - Solving Recurrence Relations by Iteration}
% % Section 5.7: 28, 30, 35*, 39
% % Note: I assigned these questions not because their answers are in the back of the book, but because the
% % problems they depend on are in the book.  Try problems 3, 5, 10 and 14, but don't spend a ton of time on
% % them.
% \begin{modenumerate}
%   \setcounter{enumi}{27}
% \item % 28
% Verify Exercise 3 by mathematical induction (fill it in!).
% \setcounter{enumi}{29}
% \item % 30
% Verify Exercise 5 by mathematical induction.
% \setcounter{enumi}{34}
% \moditem{*} % 35
% Verify Exercise 10 by mathematical induction.
% \setcounter{enumi}{38}
% \item % 39
% Verify Exercise 14 by mathematical induction.
% \end{modenumerate}
\end{document}

% LocalWords: LocalWords
