% -*- fill-column: 110 -*-
\documentclass[12pt]{article}
\usepackage[margin=1in]{geometry}
\usepackage{hyperref}
\usepackage{amsmath}
\usepackage{amsfonts}
\usepackage{amssymb}
\usepackage{amsthm}
\usepackage[margin=1in]{geometry}
% \usepackage{apacite}
\usepackage{color}
%\usepackage{sagetex}
\usepackage{fancyhdr}
\usepackage{setspace}
\pagestyle{fancy}
\lhead{\footnotesize Proof that $a^{2} + a$ must always be even}
\rhead{\footnotesize September 13 2013 -- MA325 -- Shae Erisson and
  Alyson Mavromat}

\newcommand*\diff{\mathop{}\!\mathrm{d}}
\begin{document}

\section*{Proof: $a^{2} + a$ is even}
Algebraically, this can be broken down to $a(a + 1)$. In the case that
$a$ is an odd number, then $a + 1$ will be even.

In the case that $a$ is an even number, then $a + 1$ will be odd.

In either of those cases an odd number will be multiplied by an even
number, resulting in a number with 2 as one of its factor.

At that point we can refer back to Definition 1 where an integer is
even  if there exists an integer $k$ such that $n = 2k$. $\square$
\end{document}

% LocalWords: LocalWords differentiable
