% -*- fill-column: 110 -*-
\documentclass[12pt]{article}
\usepackage[margin=1in]{geometry}
\usepackage{hyperref}
\usepackage{amsmath}
\usepackage{amsfonts}
\usepackage{amssymb}
\usepackage{amsthm}
\usepackage[margin=1in]{geometry}
% \usepackage{apacite}
\usepackage{color}
%\usepackage{sagetex}
\usepackage{fancyhdr}
\usepackage{setspace}
\pagestyle{fancy}
\lhead{\footnotesize Exercise Set 3.1}
\rhead{\footnotesize September 9 2013 -- MA325 -- Shae Erisson}

\newcommand*\diff{\mathop{}\!\mathrm{d}}
\begin{document}

\section*{Exercise Set 3.1}
\begin{enumerate}
% Section 3.1: 1, 6, 11, 12, 18, 33
\item % 1
  \begin{enumerate}
  \item False, there is not a red animal in the menagerie.
  \item True, every animal in the menagerie is either a bird or a mammal.
  \item False, there is a blue bird in the menagerie.
  \item True, there is a dog in the menagerie.
  \item False, there are blue birds in the menagerie.
  \item True, there is at least one black bird, black cat, and black
    dog in the menagerie.
  \end{enumerate}
\setcounter{enumi}{5} % 6
\item
  $R(m,n)$ is the predicate ``If $m$ is a factor of $n^{2}$ then $m$
  is a factor of $n$.
  \begin{enumerate}
  \item While 25 is a factor of 100, 25 is not a factor of 10.
  \item A small amount of Haskell code was written to discover numbers
    that have this property:
\begin{verbatim}
prop_Fail m n = prereq ==> mod n m == 0
          where prereq = n > 0 && m > 0 && mod (n*n) m == 0
\end{verbatim}
Several pairs were found, including 12 and 8, 8 and 4, and 9 and 3.
\item In the case of 5 and 10, the first number does divide both the
  number and its square.
\item Any number that evenly divides some other number will also
  divide the square of that other number, thus values are available,
  including 2 and 4, and 7 and 14.
  \end{enumerate}

\setcounter{enumi}{10} % 11
\item The simplest example is $1 \times 1 \leq 1 + 1$.
% 12
\item Almost any number works for a counter example: $\sqrt{4} \neq
  \sqrt{2} + \sqrt{2}$ since the result becomes $4 = \sim 3.8$.
\setcounter{enumi}{17} % 18
\item
  \begin{enumerate}
  \item $\exists s\in$ students, such that $s$ $E(s)$ and $M(s)$.
%  \item $\forall$
  \end{enumerate}

\setcounter{enumi}{32} % 33
\item
  \begin{enumerate}
  \item True, any number greater than 2 will be greater than 1.
  \item True, the square of any number greater than 2 will be greater than 4.
  \item True, if the square of a number is greater than 4, the number will be greater than 2.
  \item True, if the absolute value of a number is greater than 2, its square will be greater than 4, and if
    the square of a number is greater than 4, it will be greater than 2.
  \end{enumerate}

\end{enumerate}

\end{document}

% LocalWords: LocalWords differentiable
