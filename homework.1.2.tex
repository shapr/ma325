% -*- fill-column: 110 -*-
\documentclass[12pt]{article}
\usepackage[margin=1in]{geometry}
\usepackage{hyperref}
\usepackage{amsmath}
\usepackage{amsfonts}
\usepackage{amssymb}
\usepackage{amsthm}
\usepackage[margin=1in]{geometry}
% \usepackage{apacite}
\usepackage{color}
%\usepackage{sagetex}
\usepackage{fancyhdr}
\usepackage{setspace}
\pagestyle{fancy}
\lhead{\footnotesize Exercise Set 1.2}
\rhead{\footnotesize August 26 2013 -- MA325 -- Shae Erisson}

\newcommand*\diff{\mathop{}\!\mathrm{d}}
\begin{document}

\section*{Exercise Set 1.2} \footnote{Would it make grading easier for
you if the original question were included above the answers?}
\begin{enumerate}
% 1
\item Sets A and C are equal, and sets B and D are equal.
  \setcounter{enumi}{3}
\item % 4
  \begin{enumerate}
  \item Yes, $2 \in \{2\}$.
  \item The set $\{2,2,2,2\}$ has one element.
  \item The set $\{0,\{0\}\}$ has two elements.
  \item Yes, $\{0\} \in \{\{0\},\{1\}\}$.
  \item No, $0 \not\in \{\{0\},\{1\}\}$.
  \end{enumerate}

  \setcounter{enumi}{6}
\item % 7
  \begin{enumerate}    
  \item $\{-1,1\}$ No matter what power it is raised to, -1 can only result in -1 or 1.
  \item $\{0,2\}$ As in the previous question, -1 raised to any power can only be -1 or 1. When 1 is added to
    that, it can only result in either 0 or 2.
  \item $ \emptyset $ since both conditions must be true, all integers are excluded.
  \item $ \mathbb{Z} $ since either condition can be true, this is all integers.
  \item $ \emptyset $ since both conditions must be true, all integers are excluded.
  \item $ \mathbb{Z} $ since either condition can be true, this is all integers.
  \end{enumerate}

\setcounter{enumi}{7}
\item % 8
  \begin{enumerate}
  \item No, $B \not\subseteq A$, because $j$ is in $B$ but not in $A$.
  \item Yes, $C \subseteq A$, all items in $C$ are also in $A$.
  \item Yes, $C \subseteq C$, all items in $C$ are in $C$.
  \item Yes, $C$ is a proper subset of $A$, as all items in $C$ are in $A$.
  \end{enumerate}

\setcounter{enumi}{9}
\item % 10
  \begin{enumerate}
  \item No, the order of operations does matter thus the ordered pairs are not identical.
  \item No, order is significant in an ordered pair, this is not a set.
  \item Yes, the value of both ordered pairs reduces to the same ordered pair.
  \item Yes, the value of both ordered pairs reduces to the same ordered pair.
  \end{enumerate}
\end{enumerate}

\end{document}

% LocalWords: LocalWords differentiable
